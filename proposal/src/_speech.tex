\section{Phonetic Perception}
\label{sec:phonemes}

\subsection{Mice Can Learn Phonetic Categories}

I started my work by teaching mice to generalizably discriminate between two sets of naturally produced pitch-shifted consonant-vowel pairs \cite{saundersMiceCanLearn2019}. The idea was to establish phonetic perception in mice as a model that would be useful for twin problems in auditory neuroscience and phonetics: Auditory neuroscientists need better models to move beyond simple stimuli like tone pips and noise to understand how the mammalian auditory system processes the complexity that defines natural sound environments. The long history of phonetics research provides auditory neuroscience with a rich body of work, largely ungrounded in neurophysiological observation, that can inform our experimental designs and structure our predictions. Reciprocally, animal models can augment phonetic research by isolate the specifically auditory component of phonetic perception from the syntactic, semantic, motor, and social influences intrinsic to human perception; and mice provide access to observe and perturb neurophysiological processes unavailable in humans.

\begin{done}
The experiment consisted of a two-alternative forced choice task in an arena with three water ports with photosensors to detect when they poked their nose in them. The mice would poke their nose in the center port to hear a consonant-vowel (/CV/) pair beginning with a /b/ or /g/, and then were rewarded if they poked their nose in the correct flanking port corresponding to either consonant. They progressed through several shaping stages that expanded the number of possible /CV/ tokens, adding new speakers and vowels, until they could reliably categorize 20 tokens. The mice then were tested in a generalization task where on a small subset of trials the mice were presented novel tokens. We characterize generalization as the measure of having learned an abstract or at least adaptive notion of the consonant structure, rather than overlearning the training set. We trained two cohorts of mice with different sets of speakers to test whether and how strongly that patterned their responses (vs. learning some "universal" means of categorizing the tokens).

The primary analysis in the paper used a generalized linear mixed model with a logistic link function to predict binary correct/incorrect responses using the type of token (training tokens, generalization tokens, etc.) as a fixed effect nested within each mouse as a random effect. From this we estimated the relative difficulties of different kinds of generalization (eg. to novel speakers vs. novel vowels), and differences in accuracy patterns across mice and cohorts. We tested the predictions of one neurolinguistic model ("Locus equations") and found no behavioral support for them. Finally we lesioned auditory cortex and found that it returned the mice to chance accuracy -- indicating that cortex is at least involved in the computation.
\end{done}

\subsection{Models \& Mechanisms}

This work bookends my PhD, and will be in collaboration with several others in my lab. I will be working on formal modeling of the problem faced by the auditory system, and using the tools I have made to perform a version of the prior experiment such that (a proxy for) the neural activity of the mice can be observed across the learning process. I won't be able to see this project through to completion due to time constraints, but will describe what I will contribute to the work.

\begin{todo}
test

> Describe modeling work

> Describe behavior task

> Eventually am going to help with analysis and writing.
\end{todo}

