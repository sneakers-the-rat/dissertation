%!TEX root=./_preamble.tex
\section{Phonetic Perception}
\label{sec:phonemes}
\subsection{Mice Can Learn Phonetic Categories}

I started my work by teaching mice to generalizably discriminate between two sets of naturally produced pitch-shifted consonant-vowel pairs (Published in Journal of Acoustical Society of America \cite{saundersMiceCanLearn2019}). The idea was to establish phonetic perception in mice as a model that would be useful for twin problems in auditory neuroscience and phonetics: Auditory neuroscientists need better models to move beyond simple stimuli like tone pips and noise to understand how the mammalian auditory system processes the complexity that defines natural sound environments. The long history of phonetics research provides auditory neuroscience with a rich body of work, largely ungrounded in neurophysiological observation, that can inform our experimental designs and structure our predictions. Reciprocally, animal models can augment phonetic research by isolating the specifically auditory component of phonetic perception from the syntactic, semantic, motor, and social influences intrinsic to human perception; and mice provide access to observe and perturb neurophysiological processes unavailable in humans.

\begin{done}
The experiment consisted of a two-alternative forced choice task in an arena with three water ports with photosensors to detect when they poked their nose in them. The mice would poke their nose in the center port to hear a consonant-vowel (/CV/) pair beginning with a /b/ or /g/, and then were rewarded if they poked their nose in the correct flanking port corresponding to either consonant. They progressed through several shaping stages that expanded the number of possible /CV/ tokens, adding new speakers and vowels, until they could reliably categorize 20 tokens. The mice then were tested in a generalization task where on a small subset of trials the mice were presented novel tokens. We characterized generalization as the measure of having learned an abstract or at least adaptive notion of the acoustic structure of the leading consonants, rather than overlearning the training set. We trained two cohorts of mice with different sets of speakers to test whether and how strongly the particular history of their training patterned their responses (vs. learning some "universal" means of categorizing the tokens).

The primary analysis in the paper used a generalized linear mixed model with a logistic link function to predict binary correct/incorrect responses using the type of token (training tokens, generalization tokens, etc.) as a fixed effect nested within each mouse as a random effect. From this we estimated the relative difficulties of different kinds of generalization (eg. to novel speakers vs. novel vowels), and differences in accuracy patterns across mice and cohorts. We tested the predictions of one neurolinguistic model ("Locus equations") and found no behavioral support for them. Finally we lesioned auditory cortex and found that it returned the mice to chance accuracy -- indicating that cortex is at least involved in the computation.\footnote{This work also led to an interesting \href{https://www.blackhat.com/us-19/briefings/schedule/\#detecting-deep-fakes-with-mice-14467}{side avenue} talking with some cybersecurity people about using mouse models for improving deepfake detection technology. Go figure.}
\end{done}

\subsection{Models \& Mechanisms}

Given our behavioral results, the next step, had I not taken the long way around by building Autopilot, would be to see how the brain learns and processes the phoneme pairs. Systems neuroscience is in the midst of a quiet methodological upheaval, moving beyond characterizing the properties of individual neurons sampled in isolation to modeling the brain as a dynamic system whose population activity cannot be extrapolated from its individual components. This move pairs nicely with lessons learned from methodological crises in Psychology and related fields that emphasize iterative formal modeling to inform experimental design and interpretation (eg. \cite{vanrooijTheoryTestHow2021}). In this case, we can develop formal models to evaluate observed population activity in auditory cortex by working from the broad and multidisciplinary history of phonetic perception research.

Historically, the work of the Haskins lab (eg. \cite{schertzPhoneticCueWeighting2020}), and their acoustic "cue discovery" paradigm (see \cite[p.~51]{ohalaGuideHistoryPhonetic1999}) characterized auditory processing as some mapping function 

\begin{equation}
\label{eqn:map}
M = f: \mathbf{S} \to \mathbf{P}
\end{equation} 

from some stimulus space $\mathbf{S}$ to a fixed\cite{Liberman1985a} perceptual space $\mathbf{P}$ composed of innate auditory "cues." The auditory act of phonetic perception, then, is determining the phoneme category $c_s \in \mathbf{C}$, the set of possible phonemes in a language, given some perceptual representation $\mathbf{p}$ such that

\begin{equation}
\label{eqn:pfroms}
\mathbf{p} = M(\mathbf{s})
\end{equation}

\begin{equation}
\label{eqn:infer}
c_s = max( \{ p(c_i | \mathbf{p}) : c_i \in \mathbf{C} \})
\end{equation}

Finding no set of cues that uniquely identifies a phoneme, they concluded that phonetic perception cannot be purely auditory and instead arrive at a perceptual theory that relies on simulating the motor activity of an utterance\cite{Liberman1985a}.

Each of the components of this very simple model has been problematized in parallel branches of research: developmental psychologists like Patricia Kuhl studying infant speech acquisition show the clear plasticity of the auditory system \cite{kuhlEarlyLanguageAcquisition2004}, describing a "warped" perceptual space $\mathbf{P}$ that takes advantage of the "big cuts" of the mammalian auditory system's sensitivity to basic sound features like frequency and energy changes. Learning a language's phonemes could then be characterized as learning some weight vector $\mathbf{w}$ as a function $W$ of the history of perceptual representations of speech exposure $\mathbf{p}$ and the category structure of the language $\mathbf{C}$ that maximizes the separability of the phonetic categories in perceptual space. 

\begin{equation}
\label{eqn:w}
\mathbf{w} = W(\mathbf{p}, \mathbf{C})
\end{equation}

\begin{equation}
\label{eqn:infer_2}
c_s = max\big( \big\{ p(c_i | \mathbf{p} \cdot \mathbf{w}) : c_i \in \mathbf{C} \big\}\big)
\end{equation}

Cognitive psychologists like Eleanor Rosch studying the nature of cognitive categories emphasize their ill-definedness, that they resemble a Wittgensteinian "game" that does not require a neat mapping from a perceptual space to phonetic category identity\cite{roschWittgensteinCategorizationResearch1987, roschFamilyResemblancesStudies1975}. Rather than a "flat" cue space where all cues are evaluated simultaneously, phonologists have described a hierarchy of contrastive features where "lower" cues are evaluated conditionally on the value of "higher" cues\cite{Dresher2008}. Another perspective dismisses the notion of cues altogether, focusing on the informational content in the temporal dynamics of the whole speech signal\cite{kluenderLongstandingProblemsSpeech2019}.

These theories are more compatible with the deeply nonlinear computation of the brain than a simple linear evaluation of cues. The neuroscientific perspective emphasizes the active nature of perception, where the auditory system is not some passive filterbank, but a plastic organ that adapts over multiple timescales to the statistics of its input\cite{angeloniContextualModulationSound2018, holtTemporallyNonadjacentNonlinguistic2005}. This shifts the focus from learning some pre-existing cues to the process by which the brain adapts to create a joint neural-perceptual space that can serve as a flexible basis set for phonetic categorization. 

\begin{todo}
\label{todo:speech}
Though I won't be able to see this experiment through to its completion, in my last months here, I will be working with my labmates to start mesoscopic calcium imaging, measuring a proxy for neural activity, across the course of learning the phonetic discrimination task above. Imaging across the course of learning will let us not only model the computation of phonetic perception as it happens, but how the brain learns to support it. I will be responsible for developing the hardware and software to perform the behavior in a head-fixed context, formalizing our models and analytical techniques by reading across the array of disciplines that have studied phonetic perception, and then will be leaving the imaging and data collection to my labmates Sam Mehan and Rocky Penick.
\end{todo}
