%!TEX root=./_preamble.tex

\section{Infrastructure}
\label{sec:infrastructure}

Through my work on Autopilot and PVP I started pulling the thread of digital scientific infrastructure at large --- and, dear reader: all is not well. Despite a broadly\footnote{though usually passively} held whiggish view that scientific infrastructure will, in fits and starts, get "better" over time, we are circled on all sides by information conglomerates intent on carving the basic practice of science into a series of surveilled platforms that will make us long for the `simpler days' of merely being extorted for publication fees\cite{pooleySurveillancePublishing2021} (and see: \cite{zuboffBigOtherSurveillance2015}). We all want to see science made freely available, to be able to collaborate openly, to be able to make cumulative progress on shared tools and research questions. Yet one needs look no further than the judo-flip scam of open access requirements\footnote{like our own: \url{https://senate.uoregon.edu/senate-motions/us2021-18-open-access-scholarship-policy}}, where the publishing giants were able to transmute our noble intentions into doubling and tripling publication fees in their fastest growing market segment\footnote{RELX, parent of Elsevier, describes as one of their strategic priorities "grow article volume through new journal launches, the expansion of open access journals" and describes them as their major driver of revenue growth\cite{RELXAnnualReport2020}. The new plan is to charge individual authors a lot to publish, shifting cost burdens from institutions (pitting scientists against librarians against institutions to prevent something like the California University System's bargaining with them from happening again), provide a suite of platformatized tools riddled with spyware, and then repackage our personal data to their dozens of waiting markets --- scientific publication is less than a third of their business.}, to see how it is far from trivial to answer the eternal question: "how do we get there from here?" 

This is a pandisciplinary problem that needs all hands on deck: carrying on with science as normal, playing into their faustian bargain by publishing in prestige journals ostensibly for the benefit of trainees, and the attitude that it is "not our job" to take care of the health of the scientific infrastructure is exactly how we got here. Through decades of paying exorbitant publication fees, scientists have created one of the more dangerous surveillance conglomerates in the world that tracks researchers across their workflow, collects personal information like purchases, location, credit and loan applications, and funnels it all straight to insurance companies and government agencies like ICE\cite{biddleLexisNexisProvideGiant2021}. The future is grim: Elsevier already \href{https://www.scival.com/landing}{sells a product} built on workflow surveillance to employers and funding agencies to rank scientists on their productivity and how "trending" their research is, a recipe for supercharged hype cycles that benefit no one but publishers. The NIH's \href{https://datascience.nih.gov/strides/}{STRIDES} program has paid Amazon Web Services, Google Cloud, and Microsoft Azure nearly \href{https://reporter.nih.gov/search/TUZ2dhR4OEyc2XbE5FSW8g/projects}{\$100 million} to train scientists to use their grant funding to pay them more\footnote{STRIDES does not provide cloud services, just training and discounts} to store their data on their servers --- locking them in to perpetual payment.

To counter the encroachment of surveillance and platformatization we need to work collectively against concentrated and powerful organizations. To work collectively we need to be able to integrate counter-organizing into existing scientific practices, which requires a combination of technological and social development --- with heavier emphasis the latter than the former. To know what to develop and how, we need a \textit{plan}.

\begin{done}
Over the last year I have been working on \href{https://jon-e.net/infrastructure/}{a practical blueprint} for developing decentralized infrastructure to displace the publishing oligopolies and data barons that goes beyond the typical calls for new platforms and journals. This work is a collage of techniques and disciplines: merging \href{https://en.wikipedia.org/wiki/Science_and_technology_studies}{science and technology studies}, information science, investigative journalism, internet archaeology, software engineering, interface design, and so on into a story about how we got here and what we can do about it. I have traced two decades of the history of digital infrastructural development and many of its subcommunities, projects, philosophies, and technologies that shape our contemporary ecosystem. I have been reverse engineering the mechanisms of surveillance and control in our everyday technologies like \href{https://twitter.com/json_dirs/status/1486120144141123584}{the humble PDF}\cite{franceschi-bicchieraiAcademicJournalClaims2022} and publisher websites including their "\href{https://twitter.com/json_dirs/status/1466951017459716096}{enhanced readers}"\cite{deferCommentEditeursScientifiques2022} to raise the alarm that surveillance publishing is already here and it affects everyone. I have been meeting and trading ideas with a growing group of people across disciplines that typically have little contact to reimagine scientific infrastructure development as a decentralized process that can be built from the "bottom up" to displace the publishers by rendering them irrelevant, rather than competing with them in their own system. Far from a utopian pipe dream that fixes the problem with one silver bullet, what I describe is a way to repurpose existing technologies and integrate them with existing practice gradually, each component reinforcing the others. 
\end{done}

\begin{todo}
\label{todo:infrastructure}
The manuscript is still unfinished, and needs a reasonable amount of editing and polishing before it's ready for releasing. I intend to build some of the linked data tools I describe in the work into the work itself to let different audiences find the information they're interested in to make reading it a bit more manageable\footnote{The document as it stands is quite long}. The next phase is a bit of an experiment in open publishing: The document is built such that it automatically compiles to both HTML and PDF from a git repository, and I'm going to see if it's possible to have open collaboration on a living scientific document. Before I release it publicly, I plan on inviting many of the folks I've been talking to about this for comments, edits, additions, and anyone who submits a pull request becomes a co-author. Whenever the paper is updated, I will see if it's within BioRxiv's terms of service to automatically upload a new version.
\end{todo}
