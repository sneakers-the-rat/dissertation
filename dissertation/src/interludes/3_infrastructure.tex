\chapter*{When Tools Become Infrastructure}

As I spent more time in the tool-building world, getting a broader sense of the way science was done across disciplines, the feeling that I had about our broken experimental tooling that led to Autopilot started to creep into more domains. 

I had been getting more interested in internet technologies, formats, wikis and loosely-structured metadata in the course of my work, and was constantly faced with the mismatch between what was possible for scientific communication and the way it was constrained by the publishing system. The interrelationship between the various forms of digital infrastructure (or lack thereof) and the broader structuring forces in science was harder and harder to ignore, and burst from me like a flood in one department meeting focused on collaboration. It felt so \textit{sad} to me that all my brilliant colleagues simply couldn't collaborate effectively because of the lack of basic tools. The scientific project as a whole started to feel \textit{doomed} if we were to continue working this way, individual labs slicing off tiny slivers of reality to look at alone, in static papers that were already years out of date by the time they were published. 

What started as a sketch of infrastructure for the sake of making science \textit{easier} quickly turned into one for making science \textit{ethical} as it became clear that the primary limitations were not technical, but social, and designed specifically to enable profit to be extracted at every point. I started to expand my scope, using some techniques from investigative journalism I had picked up from a few side-projects to dig into the market structure of the surrounding industries and the plans they had for the future of science. The story kept getting bleaker, but as I went I kept stumbling across and embedding myself within digital diasporic communities that had been dreaming of better infrastructure for almost as long as the internet had existed. This also merged with some lessons in social organization that I had picked up from my coop and union, and so in the twilight of the "open science" movement, sputtering along, chasing its tail, it seemed like the main thing we needed was a new vision for organizing something genuinely transformative rather than nibbling around the edges of "openness" while the information giants ate the rest. That might explain the perhaps uneven tone of \textit{terror} and \textit{rage} but also \textit{hope} that runs through this piece. 

The lid is off, the box is open, and for the foreseeable future I don't think I'll be able to return to basic research until we make some substantive progress in changing how it's done. Rather than abandoning science altogether though, I remain convinced we can fix some of its worst problems if enough of us believe it is possible and are willing to take the risks and do the work to make it happen --- and who knows, I'm naïve enough to believe that in the process we might start taking those "broader impact" sections seriously and build something that genuinely benefits society at large.