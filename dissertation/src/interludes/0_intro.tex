%!TEX root = ../_preamble.tex

\chapter*{Introduction}

This is a piece about the power of swarms.

I arrived in neuroscience after bottoming out in politics and economics as I learned that studying the emergent dynamics of belief and power was not, in fact, what those academic disciplines studied. Somehow, my intellectual history professor Bill Duvall was able to identify that my rudderless interests were pointing towards neuroscience, and somehow he was right.

I was blessed with spending my first years outside of the canonical training of neuroscience with Emma Coddington, who was more interested in the multiscale interplay between neural, endocrine, and emotional systems than the reductive filter-bank model of the brain. In neuroscience I found an endless abyss of systems creating themselves, loose boundaries between layers of organization and chaos. I still find myself in the fringes, thinking about the brain in its nonlinear dynamics, immune to averaging and estimation, an organ with its own ideas about its activity without a clear "code" or purpose aside from its own persistence. It is all the noise and interdependence and local organization that results in some messy superstructure that keeps me near it.

It has been impossible for me to ignore the systems that structure the practice of science long enough for me to spend much time doing it, though. The same things that draw me towards studying the brain make me look upward at the messy, anarchic processes that emerge as science --- and the higher-order structuring forces that condition it. We are all little neurons, only aware of our immediate n-depth neighbors against the backdrop of the structures that our local awareness creates. The reality of swarms is the slippery interdependence that merges that local autonomy with the overriding circumstance that binds them together. The miracle of the brain is the miracle of society, how the blend of autonomy and independence makes something more spectacular than its parts. The challenge for understanding both is valuing the messiness and unplannedness of their agents alongside their necessary interdependence, without which they would lose meaning.

"Rough consensus and running code," cribbed from internet protocol architects\citep{clarkCloudyCrystalBall1992}, captures the tautology that what works is whatever works. Neither a grand planned architecture nor a libertarian focus on the disconnected autonomy of its agents describes systems capable of emergent behavior. What works is a fluid and evolving consensus based on the agents organizing together to meet their needs. This is the thread that binds my work: ill-defined categories computed by neural assemblies, loose design in decoupled systems in experimental tools, and the linked interoperability of digital infrastructure.  It's not clear to me whether this is an anarchist's view of the brain, or a neuroscientist's view of politics, and it's not necessarily important to me to resolve that.

These first two pieces focus on the ability for auditory cortex to learn from continuous sounds to create the ill-defined perceptual categories of phonemes, the initial plan for my work that was quickly interrupted.



