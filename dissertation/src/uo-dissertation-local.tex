%!TEX root = ./_preamble_institutional.tex
% ---------
% Compatibility with existing classes
\newenvironment{acknowledgements}{}{}
\newenvironment{fullwidth}{\restoregeometry}{}
% \newenvironment{leftbar}{\begin{quote}}{\end{quote}}
\newenvironment{marginfigure}[1][]{\begin{figure}[h]}{\end{figure}}
\newenvironment{margintable}[1][]{\begin{table}}{\end{table}}

\let\citep\cite
% \let\sidenote\footnote
\newcommand{\sidenote}[2][1]{\footnote{#2}}
\let\marginnote\footnote
\newcommand{\newthought}[1]{#1}
\newcommand{\mainmatter}{}

% --------
% Additional Packages
\usepackage{tabularx}
\usepackage{mdframed}
\usepackage[cache=true]{minted}
\usepackage{enumitem}

\usepackage{listings}
\usepackage{tikz}
\usepackage{fontspec}

\usepackage{placeins}
\usepackage{minibox}

% ----------
% Local imports

\usepackage{color}
\usepackage{fancyvrb}
\newcommand{\VerbBar}{|}
\newcommand{\VERB}{\Verb[commandchars=\\\{\}]}
\DefineVerbatimEnvironment{Highlighting}{Verbatim}{commandchars=\\\{\}}
% Add ',fontsize=\small' for more characters per line
\newenvironment{Shaded}{}{}
\newcommand{\AlertTok}[1]{\textcolor[rgb]{1.00,0.00,0.00}{\textbf{#1}}}
\newcommand{\AnnotationTok}[1]{\textcolor[rgb]{0.38,0.63,0.69}{\textbf{\textit{#1}}}}
\newcommand{\AttributeTok}[1]{\textcolor[rgb]{0.49,0.56,0.16}{#1}}
\newcommand{\BaseNTok}[1]{\textcolor[rgb]{0.25,0.63,0.44}{#1}}
\newcommand{\BuiltInTok}[1]{#1}
\newcommand{\CharTok}[1]{\textcolor[rgb]{0.25,0.44,0.63}{#1}}
\newcommand{\CommentTok}[1]{\textcolor[rgb]{0.38,0.63,0.69}{\textit{#1}}}
\newcommand{\CommentVarTok}[1]{\textcolor[rgb]{0.38,0.63,0.69}{\textbf{\textit{#1}}}}
\newcommand{\ConstantTok}[1]{\textcolor[rgb]{0.53,0.00,0.00}{#1}}
\newcommand{\ControlFlowTok}[1]{\textcolor[rgb]{0.00,0.44,0.13}{\textbf{#1}}}
\newcommand{\DataTypeTok}[1]{\textcolor[rgb]{0.56,0.13,0.00}{#1}}
\newcommand{\DecValTok}[1]{\textcolor[rgb]{0.25,0.63,0.44}{#1}}
\newcommand{\DocumentationTok}[1]{\textcolor[rgb]{0.73,0.13,0.13}{\textit{#1}}}
\newcommand{\ErrorTok}[1]{\textcolor[rgb]{1.00,0.00,0.00}{\textbf{#1}}}
\newcommand{\ExtensionTok}[1]{#1}
\newcommand{\FloatTok}[1]{\textcolor[rgb]{0.25,0.63,0.44}{#1}}
\newcommand{\FunctionTok}[1]{\textcolor[rgb]{0.02,0.16,0.49}{#1}}
\newcommand{\ImportTok}[1]{#1}
\newcommand{\InformationTok}[1]{\textcolor[rgb]{0.38,0.63,0.69}{\textbf{\textit{#1}}}}
\newcommand{\KeywordTok}[1]{\textcolor[rgb]{0.00,0.44,0.13}{\textbf{#1}}}
\newcommand{\NormalTok}[1]{#1}
\newcommand{\OperatorTok}[1]{\textcolor[rgb]{0.40,0.40,0.40}{#1}}
\newcommand{\OtherTok}[1]{\textcolor[rgb]{0.00,0.44,0.13}{#1}}
\newcommand{\PreprocessorTok}[1]{\textcolor[rgb]{0.74,0.48,0.00}{#1}}
\newcommand{\RegionMarkerTok}[1]{#1}
\newcommand{\SpecialCharTok}[1]{\textcolor[rgb]{0.25,0.44,0.63}{#1}}
\newcommand{\SpecialStringTok}[1]{\textcolor[rgb]{0.73,0.40,0.53}{#1}}
\newcommand{\StringTok}[1]{\textcolor[rgb]{0.25,0.44,0.63}{#1}}
\newcommand{\VariableTok}[1]{\textcolor[rgb]{0.10,0.09,0.49}{#1}}
\newcommand{\VerbatimStringTok}[1]{\textcolor[rgb]{0.25,0.44,0.63}{#1}}
\newcommand{\WarningTok}[1]{\textcolor[rgb]{0.38,0.63,0.69}{\textbf{\textit{#1}}}}


% ----
% Package Options

\hypersetup{%
	colorlinks = true,
	urlcolor = DarkOrchid,
	citecolor = DarkOrchid,
	backref= section,
  pdftitle={Swarmpunk: Rough Consensus and Running Code in Brains, Machines, and Society},
  pdfkeywords=Dissertation,
	pdfauthor=Jonny L Saunders,
	pdfcreator={Jonny L Saunders}
}

\usetikzlibrary{tikzmark}
\usetikzlibrary{calc}
\usetikzlibrary{positioning}
\usetikzlibrary{decorations.pathreplacing}

% ----
% Environments

\renewenvironment{abstract}{%
\vspace{1em}
\begin{mdframed}[
     linewidth=1pt,
     linecolor=black,
     bottomline=false,topline=false,rightline=false,
     innerrightmargin=0pt,innertopmargin=0pt,innerbottommargin=0pt,
     innerleftmargin=1em,% Distance between vertical rule & proof content
     skipabove=0\baselineskip
   ]
}{%
\end{mdframed}
}

\newcommand{\linkfigure}[3]{\begingroup
\setbox0=\hbox{\includegraphics[height=0.75cm,keepaspectratio]{#1}}%
\parbox{\wd0}{\box0}\endgroup \hspace{6pt} \href{#3}{\sffamily\large\textcolor{red}{#2}}}

% right aligned tabularx
\newcolumntype{R}{>{\raggedleft\arraybackslash}X}

% Styling for code blocks
\usemintedstyle{colorful}
%\setminted{fontsize=\small}

% set escape chars in listing
\lstset{
    language = Python,
    breaklines=true,
    escapeinside=||
}

\definecolor{aliceblue}{rgb}{0.98, 0.98, 0.99}


\newminted{python}{
  linenos=true,
  frame=single,
  framerule = 1pt,
  labelposition = topline,
  bgcolor=aliceblue,
  escapeinside=||
}

\newminted{matlab}{
  linenos=true,
  frame=single,
  framerule = 1pt,
  labelposition = topline,
  bgcolor=aliceblue,
  escapeinside=||
}

% ------------------
% Tufte mimics

\newsavebox{\marginfloatbox}
\newenvironment{marginfloat}[2][-1.2ex]%
  {\FloatBarrier% process all floats before this point so the figure/table numbers stay in order.
  \begin{lrbox}{\marginfloatbox}%
  \begin{minipage}{\marginparwidth}%
    % \@tufte@caption@font%
    \def\@captype{#2}%
    \hbox{}\vspace*{#1}%
    % \@tufte@caption@justification%
    % \@tufte@margin@par%
    \renewcommand{\baselinestretch}{1}
    \noindent%
  }
  {\end{minipage}%
  \end{lrbox}%
  \marginpar{\usebox{\marginfloatbox}}%
  \renewcommand{\baselinestretch}{1.5}%
  }

% \setlength{\abovecaptionskip}{3pt plus 2pt minus 3pt}
% \setlength{\belowcaptionskip}{0pt}
\captionsetup[lstfloat]{skip=0pt}

\lstset{aboveskip=0pt,belowskip=0pt}
% \setlength\intextsep{2pt}



% ------------------
% Typography

\setmainfont{EBGaramond}[
	Path = fonts/garamond/ ,
	Extension = .ttf ,
	UprightFont = *-Regular ,
	BoldFont = *-Bold ,
	ItalicFont = *-Italic ,
	BoldItalicFont = *-BoldItalic
]
% \setmainfont{Courier New}
% \setmainfont{Courier New}[
% 	Path = /System/Library/Fonts/Supplemental/ ,
% 	Extension = .ttf ,
% 	BoldFont = * Bold ,
% 	ItalicFont = * Italic ,
% 	BoldItalicFont = * Bold Italic
% ]

\setmonofont{FiraCode}[
  Path = fonts/fira/ ,
  Extension = .ttf ,
  UprightFont = *-Retina ,
  BoldFont = *-Bold ,
  Scale=MatchLowercase
]

\setsansfont{WorkSans}[
  Path = fonts/worksans/ ,
  Extension = .otf ,
  UprightFont = *-Regular ,
  BoldFont = *-SemiBold ,
  ItalicFont = *-Italic ,
  BoldItalicFont = *-SemiBoldItalic
]

% --------
% TOC 
% enable changing toc depth mid-document
% https://tex.stackexchange.com/a/395858
\newcommand{\changelocaltocdepth}[1]{%
  \addtocontents{toc}{\protect\setcounter{tocdepth}{#1}}%
  \setcounter{tocdepth}{#1}%
}


% ----
% etc.

\definecolor{RED}{HTML}{FF0000}
