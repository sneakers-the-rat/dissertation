\chapter*{Abstract}

Jonny L. Saunders

\noindent Doctor of Philosophy

\noindent Department of Psychology

\noindent August 2022

\noindent Title: Swarmpunk: Rough Consensus and Running Code in Brains, Machines, and Society

This is a love letter to the power of swarms.

The swarm is marbled within and between individual autonomy and collective organization, the uncontained anarchic multitude with multiple bodies at multiple scales, always resisting delimiting definition but everywhere acting together. The swarm is the structure of natural complexity, the narrow but persistent condition where the riotous interdependent agency of individual components emerge as something recognizably distinct than when alone. Unconcerned with reductive theories of everything, laughing at objectivist fantasies of heroic ur-autonomy, the swarm is what \textit{works}.

I trace reflections of the swarm in three interlocking systems at different scales: ill-defined phonetic categories in the brain computed by swarms of neurons, experimental tooling that merges swarms of computers and people as groupware, and the potential for swarmlike organization to rebuild the social-infrastructural systems that structure knowledge work in our age of informational capitalism. Starting with a research question drawn from neuroscience, phonetics, and philosophy, I work outwards, dispensing with crude gestures at "discipline" to find echoes of how we model the brain in how we build tools to study it --- and how we dream of better futures. In each I find the swarm, an embrace of fluid and messy consensus instead of an engineer's or ideologue's just-so systems. 

This is a message of hope for what we can build together.

This dissertation includes previously published co-authored material.